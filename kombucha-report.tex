\documentclass[12pt]{article}

\usepackage{sbc-template} 
\usepackage{graphicx,url}
\usepackage{url}
\usepackage[english]{babel} 
\usepackage[utf8]{inputenc} 
\usepackage[T1]{fontenc}
\usepackage[normalem]{ulem}
\usepackage[hidelinks]{hyperref}

\usepackage[square,authoryear]{natbib}
\usepackage{amssymb} 
\usepackage{mathalfa} 
\usepackage{algorithm} 
\usepackage{algpseudocode} 
\usepackage[table]{xcolor}
\usepackage{array}
\usepackage{titlesec}
\usepackage{mdframed}
\usepackage{listings}

\usepackage{amsmath} 
\usepackage{booktabs}

\urlstyle{same}

\newcolumntype{L}[1]{>{\raggedright\let\newline\\\arraybackslash\hspace{0pt}}m{#1}}
\newcolumntype{C}[1]{>{\centering\let\newline\\\arraybackslash\hspace{0pt}}m{#1}}
\newcolumntype{R}[1]{>{\raggedleft\let\newline\\\arraybackslash\hspace{0pt}}m{#1}}

\newcommand\Tstrut{\rule{0pt}{2.6ex}} 
\newcommand\Bstrut{\rule[-0.9ex]{0pt}{0pt}} 
\newcommand{\scell}[2][c]{\begin{tabular}[#1]{@{}c@{}}#2\end{tabular}}

\usepackage[nolist,nohyperlinks]{acronym}

\title{Preparation of Scoby mother culture and initiation of kombucha fermentation in Batticaloa, Sri Lanka.}

\author{Cristian Silva\inst{1} Saad Chinoy\inst{1} Pramodya Saumyamali \inst{2}} 


\address{Digital Naturalism Conference - 2022
	\email{cristiansilvaofficial@gmail.com}
}



\begin{document} 
	
	\maketitle
	
	\begin{abstract}
	    A fermented, mildly effervescent, probiotic beverage called kombucha is frequently ingested for its alleged health advantages. It is created through the fermentation of tea and sugar with a biofilm known as "Scoby”. Scoby is a symbiotic culture of yeast and bacteria living in biofilm made with cellulose. It is claimed to strengthen the immune system, Improve gut functions, maintain a healthy body mass index (BMI), and protect against high blood pressure and heart disease and cancer. In this project, we explored the process of making SCOBY mother culture and doing kombucha fermentation with Ceylon black tea which could be a beneficial,  locally brewed probiotic  drink for underserved local communities in Sri Lanka

	\end{abstract}
	    
		
	
	
	\section{Introduction}
	\label{sec:introduction}
	Kombucha fermentation has Asian origins. It's believed China was the first country to discover kombucha. Kombucha is fermented sweetened tea. The kombucha fermentation is done by ‘SCOBY’ also known as ‘tea fungus’.  SCOBY is a symbiotic culture of bacteria and yeast in a cellulose film and acetic acid bacteria and osmophilic yeast are the predominant microorganisms in SCOBY. 

Kombucha contained antioxidants, polyphenols, amino acids, antibiotics and organic acids such as acetic, gluconic, malic, lactic, citric and etc. Its taste is slightly acidic and carbonated which is preferred by a lot of consumers as an alternative to soft drinks, and low-alcoholic beverages. 

Due to a unique combination of bacteria, yeast and other biochemicals kombucha offer myriad health benefits for its consumers. Reduction of inflammation, blood pressure and cholesterol reduction, and reducing the risk of cancer are one of the few benefits of consuming kombucha. 

The microbial composition of SCOBY is not well defined because of the involvement of a large number of bacteria and yeast species. It can be varied due to fermentation temperature, type of herbs used, other sugar substrates and origin. Therefore, many bacteria and yeast strains involved in fermentation are still unknown.  
In this project, we have worked on establishing SCOBY mother culture in DreamSpace bio lab, and then fermentation of Kombucha with Ceylon black tea. 

	\section{Objective}
	\label{sec:objective}
	
Establish the know-how of making SCOBY mother  culture and kombucha fermentation.
	
	
	\section{Materials}
	\label{sec:materials}
	
	Ceylon black tea powder, White Sugar, Water, 500 ml glass jars, Fannel cloth, Rubber bands, Electric stove, Beakers, Spoon, Oven gloves


	
	\section{Methodology}
	\label{sec:metodology}
	
	

 
\subsection{Preparation of mother culture
}

Measure 300 ml of water and heat it in a beaker while stirring and add sugar (10\% W/V). Once sugar is dissolved take the beaker away from the stove and let it cool down to room temperature. Measure 3 g of Ceylon black tea and boil 50 ml of water. Once the water is boiling add 3 g of tea and let it cool down to room temperature. After that, mix the tea with the sugar solution and transfer the mixture to a 500 ml glass jar. Cover the jar with a flannel cloth and fix the cloth to the jar by using a rubber band. Let the jar ferment for 14 to 21 days at room temperature. Observe the SCOBY formation and remove mold from the SCOBY if necessary.



\subsection{Kombucha fermentation of Ceylon black tea }
Measure 200 ml of water and heat it in a beaker while stirring and add sugar (10\% W/V). Once sugar is dissolved take the beaker away from the stove and let it cool down to room temperature. Measure 5 g of Ceylon black tea and boil 75 ml of water. Once the water is boiling add 5 g of tea and let it cool down to room temperature. After that, mix the tea with the sugar solution and transfer the mixture to a 500 ml glass jar.  Remove the SCOBY mother culture from the previous jar and transfer it to the sweetened tea mixture and add 100 ml of already fermented tea from the mother culture jar. Cover the jar with a flannel cloth and fix the cloth to the jar by using a rubber band. Let the jar ferment for 5-7 days at room temperature. 
\subsection{ Kombucha fermentation of Avaram senna tea (Ranawara)}

Measure 200 ml of water and heat it in a beaker while stirring and add sugar (10\% W/V). Once sugar is dissolved take the beaker away from the stove and let it cool down to room temperature. Measure 5 g of Ranawara tea and boil 75 ml of water. Once the water is boiling add 5 g of tea and let it cool down to room temperature. After that, mix the tea with the sugar solution and transfer the mixture to a 500 ml glass jar.  Remove the SCOBY mother culture from the previous jar and transfer it to the sweetened tea mixture and add 100 ml of already fermented tea from the mother culture jar. Cover the jar with a flannel cloth and fix the cloth to the jar by using a rubber band. Let the jar ferment for 5-7 days at room temperature. 



        \section{Observations and Discussion}
Making mother culture was the most challenging because as we did this project for the first time we did not have a clear understanding of the time it takes SCOBY mother culture to develop on top of sweetened tea. We have figured out that using 3 to 5 g of Ceylon black tea would be ideal to introduce favourable bacteria and yeast while not hindering microbial growth with the phytochemicals of tea.  It took 2 weeks time to appear the first visible SCOBY, however soon after the appearance of SCOBY, black mold started to grow on top of SCOBY.  While trying to remove mold from SCOBY resulted breaking of the cellulose film. Initially, we used metal lids with holes covered in cotton to close the jars and we replaces lids with a flannel cloth to provide more oxygen for SCOBY to fight off the mold.

    

        \begin{figure}[ht]
\centering
\includegraphics[width=0.5\linewidth, angle= 90]{kombucha-tea.jpg}
\caption{Ceylon black tea Kombucha}
\label{fig:view}
\end{figure}

Once SCOBY culture was established approximately after 21 days, it appeared as a thick sponge. It can be used to make other SCOBY baby cultures by tearing mother cultures into small pieces and introducing them to newly made sweetened tea. 

In this project, we used herbal tea known as Ranawara tea and  Ceylon black tea to make Kombucha. Two  SCOBY mother cultures were developed and transferred to each jar. After 7 days of fermentation, we took the kombucha tea from jars and stored it in glass bottles for 4 days for the second round of fermentation. Then bottles were stored in the fridge and severed accordingly. The taste of the two kombucha tea was distinguishable and it was thicker than normal tea. 



            	
    \end{document}

